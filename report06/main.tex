% !TEX encoding = UTF-8 Unicode
\RequirePackage{fix-cm}
\documentclass[a4paper,10pt,UTF8]{paper}
%\documentclass[a4paper,10pt,UTF8]{ctexart}

\usepackage[english]{babel}
\usepackage{fancyhdr,array,lastpage,amsmath,mathtools,enumitem,graphicx,multirow,tocbibind,longtable,makecell,varwidth,titlesec,bm,booktabs,comment,minted}
\usepackage{enumitem}
\usepackage{hyperref}
\hypersetup{hidelinks}




\usepackage[left=2.54cm,right=2.54cm,top=2.54cm,bottom=2.54cm]{geometry}
\usepackage[font=footnotesize,labelfont=bf]{caption}
\usepackage{tikz,flowchart}
\usepackage{ctex}
\usepackage{xeCJK}%中文字体
\usetikzlibrary{shapes,shapes.geometric,arrows,matrix,calc}
\usetikzlibrary{circuits.logic}

% \usetikzlibrary{circuits.logic.custom}
\usetikzlibrary{circuits.logic.IEC}
\usetikzlibrary{shadows}
\usepackage{listings}
\usepackage[Q=yes]{examplep}
\usepackage{fancyhdr}
\usepackage{alphalph}
\usepackage{indentfirst}

% \setCJKsansfont{黑体}
\setmainfont{PingFang SC}
\setCJKmainfont{PingFang SC}
\setCJKsansfont{PingFang SC}
\setmonofont{Monaco}

\newenvironment{sol}
  {\par\vspace{2mm}\noindent{\bf Solution}. }

\lstset{escapeinside=``, breaklines=true, frame=none, extendedchars=false, basicstyle=\ttfamily, showstringspaces=false}


\setlength{\parindent}{2em}
\setlength{\parskip}{1.5ex plus 0.5ex minus 0.2ex}
\linespread{1.1}

\bibliographystyle{plain}

\numberwithin{equation}{section}
\numberwithin{figure}{section}


\setcounter{secnumdepth}{3}
\setcounter{tocdepth}{3}

\title{华东师范大学计算机科学技术系上机实验报告}

\begin{document}
\pagestyle{fancy}
\chead{\small\color{gray}华东师范大学计算机科学技术系上机实验报告}
\lhead{}
\rhead{}
\makeatletter
\def\headrule{{\if@fancyplain\let\headrulewidth\plainheadrulewidth\fi%
\color{gray}\hrule\@height 0.2pt\@width\headwidth}
  \vspace{6mm}}
\makeatother

\newcommand{\HRule}{\rule{\linewidth}{1mm}}
\newcommand{\dai}{\textbf{Dais-CMX16$^+$}}

{\center {\huge \bfseries \LARGE{华东师范大学计算机科学技术系上机实验报告}} \\ [0.8cm]

\small{
  \begin{minipage}[t]{.32\linewidth}
    \textbf{课程名称:}嵌入式系统原理与实践\\
    \textbf{指导教师:}沈建华\\
    \textbf{上机实践名称:} 存储器映射与读写\\
    \textbf{实践编号:}实验 6
  \end{minipage}
  \begin{minipage}[t]{.32\linewidth}
    \textbf{年级:}17 级\\
    \textbf{姓名:}朱桐\\
    \textbf{学号:}10175102111\\
    \textbf{组号:}A
  \end{minipage} 
  \begin{minipage}[t]{.32\linewidth}
    \textbf{上机实践成绩:} \\
    \textbf{创新实践成绩:} \\
    \textbf{上机实践日期:}2019/11/12\\
    \textbf{上机实践时间:}2 学时\\
  \end{minipage}
}
\HRule \\[0.5cm]
}

\definecolor{bg}{rgb}{0.95,0.95,0.95}
\newminted{asm}{bgcolor=bg}
\newminted{c}{bgcolor=bg}

\section{实验目的}

\begin{enumerate}
    \item 掌握MSP432P401R的存储器映射
    \item 掌握NOR Flash的读写操作
\end{enumerate}

\section{实验设备}

\begin{enumerate}
  \item 软件Keil5;
  \item MSP432P401R开发板
\end{enumerate}

\section{实验内容}

\begin{enumerate}
  \item 通过实验验证存储器的映射分布
  \item 学习Flash的擦除、写入和读取操作
  \item 完成课堂练习:烧录指示灯的实现
\end{enumerate}

\section{实验原理}

\subsection{存储器映射}


在嵌入式处理器内,集成了多种类型的Memory,通常我们称同一类型的Memory为一个Memory Block。一般情况下,处理器设计者会为每一个memory block分配一个数值连续的自然数集合作为该block的地址编码。这种自然数集合与block的对应关系就是Memory Map(存储器映射、内存映射),有时也叫地址映射(Address Map)。

本实验中我们主要使用Block0,Block0 主要用于设计片内的 FLASH。 程序放置在FLASH中,也就是0x0000\_0000~0x0040\_0000区域,如图 \ref{fig:1} 所示。

\subsection{Flash的擦除和写入}

Flash存储器区域主要分为:主存储器和信息存储器。主存储器和信息存储器均被等分为2个BANK:BANK0和BANK1。当程序在BANK0进行写入或擦除时,同时可以在BANK1进行读取或执行操作。

主存储器大小为256KB,由64个扇区组成,每个扇区大小为4KB,最小擦除粒度为4KB(1扇区);主存储器由两个独立的BANK组成:BANK0和BANK1,每个BANK为128KB,各有32个扇区。

信息存储器大小为16KB,由4个扇区组成,每个扇区大小为4KB,最小擦除粒度同样为4KB(1扇区);信息存储器也由两个独立的BANK组成:BANK0和BANK1,每个BANK为8KB,各有2个扇区。

\subsubsection{操作函数}

\begin{ccode}
  bool MAP_FlashCtl_protectSector(uint_fast8_t memorySpace,
   uint32_t sectorMask);
\end{ccode}

函数用途:启用指定扇区的写保护;
返回值:若写保护成功启用则返回true,反之亦然。

\begin{ccode}
  bool MAP_FlashCtl_unprotectSector(uint_fast8_t memorySpace,
   uint32_t sectorMask);
\end{ccode}

函数用途:禁用指定扇区的写保护;
返回值:若写保护成功禁用则返回true,反之亦然

\begin{ccode}
  bool MAP_FlashCtl_performMassErase(void);
\end{ccode}

函数用途:对所有未开启写保护的扇区执行整体擦除;若扇区启用写保护,则忽略;
返回值:若成功擦除则返回true,反之亦然。

\begin{ccode}
  bool MAP_FlashCtl_programMemory(void* src, void* dest,
   uint32_t length);
\end{ccode}

函数用途:用提供的数据对扇区的指定地址进行写入;
返回值:若成功写入则返回true,反之亦然。

\subsection{控制 LED 灯}

这部分内容上次实验就用过了,但是 ppt 中没有详细讲。从书中及其他资料我们得知,GPIO\{X\} 中的 PIN0,1,2 分别代表 LED\{X\} 的红灯、绿灯和蓝灯。同时开启可以有混色效果。

\section{实验步骤}

\subsection{Flash 的擦除和写入}

\begin{enumerate}
  \item 打开位于工程06\_02的代码,单击Debug按钮,进入调试状态;单击Step Over按钮,可单步调试;
  \item 程序运行到第一次擦除后,观察指定扇区的值全为 F,这是因为 Flash 擦除原理只能由 0 到 1
  \item 程序运行到第一次写入后,观察指定扇区的值全为 56,因为同时关闭了两个扇区的写保护,所有都会被写入 56
  \item 程序运行到第二次擦除后,观察指定扇区的值只有 S30 为 56,因为开启了 S31 的写保护
  \item 在重新上电后,观察指定扇区的值没有发生变化,因为 Flash 是不易失的
\end{enumerate}

\subsection{烧录指示灯的实现}

\begin{enumerate}
  \item 自行新建keil工程,工程名格式:burn\_学号,并存放在工程06\_03文件夹内;
  \item 将自己学号的后两位数字以十六进制存入变量stuNum,例如学号后两位为33,则strNum=0x33;同时请将CPU主频设置为48M; 
  \item 按压S1键,擦除主存储器BANK1的29扇区,成功擦除则使LED2的蓝灯常亮,否则使LED1(红灯)常亮;
  \item 按压S2键,将stuNum写入主存储器BANK1的29扇区,成功写入则使LED2的绿灯常亮,否则使LED1(红灯)常亮;
  \item 同时按压S1键和S2键,读取主存储器BANK1的29扇区的任意单个字节至变量S29,如果变量S29的值等于strNum,则使得LED2亮白色,否则LED1(红灯)常亮;
\end{enumerate}

\section{调试过程、结果与分析}

\subsection{Flash 的擦除和写入}

不需要对代码进行改动

\subsection{烧录指示灯的实现}

\begin{ccode}
  #define BANKl_S29 (0x3D000)
  #define strNum 0x11
  #define SIZE 4096
  #define COUNT (1200000)
\end{ccode}

分别表示 BANK1 29扇区的位置,我的学号末两位(11),扇区大小,循环次数(用来等待两个键一起按的时间差)

\begin{ccode}
  volatile char S29 = 0x00;
  volatile int pos;

  // fill the bank with strNum
  memset(array, strNum, SIZE);

  // stop the watch dog
  MAP_WDT_A_holdTimer();

  // set MCLK to 48MHz
  MAP_PCM_setCoreVoltageLevel(PCM_VCORE1);
  FlashCtl_setWaitState(FLASH_BANK0, 1);
  FlashCtl_setWaitState(FLASH_BANK1, 1);
  MAP_CS_setDCOCenteredFrequency(CS_DCO_FREQUENCY_48);

  // close the write protection
  MAP_FlashCtl_unprotectSector(FLASH_MAIN_MEMORY_SPACE_BANK1,
   FLASH_SECTOR29);

  // LED1 R(P1.0) LED2 R(P2.0) LED2 G(P2.1) LED2 B(P2.2)
  MAP_GPIO_setAsOutputPin(GPIO_PORT_P1, GPIO_PIN0);
  MAP_GPIO_setAsOutputPin(GPIO_PORT_P2, GPIO_PIN0);
  MAP_GPIO_setAsOutputPin(GPIO_PORT_P2, GPIO_PIN1);
  MAP_GPIO_setAsOutputPin(GPIO_PORT_P2, GPIO_PIN2);

  // initialize, should be off when start
  MAP_GPIO_setOutputLowOnPin(GPIO_PORT_P1, GPIO_PIN0);
  MAP_GPIO_setOutputLowOnPin(GPIO_PORT_P2, GPIO_PIN0);
  MAP_GPIO_setOutputLowOnPin(GPIO_PORT_P2, GPIO_PIN1);
  MAP_GPIO_setOutputLowOnPin(GPIO_PORT_P2, GPIO_PIN2);

  // S1(P1.1) S2(P1.4)
  MAP_GPIO_setAsInputPinWithPullUpResistor(GPIO_PORT_P1, GPIO_PIN1);
  MAP_GPIO_setAsInputPinWithPullUpResistor(GPIO_PORT_P1, GPIO_PIN4);
  MAP_GPIO_enableInterrupt(GPIO_PORT_P1, GPIO_PIN1);
  MAP_GPIO_enableInterrupt(GPIO_PORT_P1, GPIO_PIN4);
  MAP_GPIO_clearInterruptFlag(GPIO_PORT_P1, GPIO_PIN1);
  MAP_GPIO_clearInterruptFlag(GPIO_PORT_P1, GPIO_PIN4);

  // interrupt
  MAP_Interrupt_enableInterrupt(INT_PORT1);
  MAP_Interrupt_enableMaster();
\end{ccode}

初始化,总共干了以下事情

\begin{enumerate}
  \item 准备写入的数据,大小为 4096 的数据,里面全是 11 (我的学号末 2 位)
  \item 停止看门狗计数器
  \item 设置 CPU 主频为 48MHz
  \item 关闭 Bank1 S29 写保护
  \item 设置 LED1 红灯为 P1.0,LED2 红灯绿灯蓝灯分别为P2.0、P2.1、P2.2,并初始化为暗状态
  \item 设置 S1为 P1.1,S2 为 P1.4
  \item 初始化清楚中断标志并且开启中断
\end{enumerate}

\begin{ccode}
  void PORT1_IRQHandler(void)
  {
    uint32_t status;
    status = MAP_GPIO_getEnabledInterruptStatus(GPIO_PORT_P1);
    MAP_GPIO_clearInterruptFlag(GPIO_PORT_P1, status);

    // 1st bit rep for PIN1, 2nd bit for PIN4
    status = (status & GPIO_PIN1) | ((status & GPIO_PIN4) << 1);

    // update the button
    button |= status;

    // start the timer
    if (button == 0)
    {
      time = COUNT;
    }
  }
\end{ccode}

中断控制函数,获取中断标志,并且加以转化,最低位二进制表示 S1,第二个表示 S2。如果第一次按下 S1 或 S2 则开启计数。如果两下都在一定的时间内,这里是空循环跑 COUNT 次,则看作是同时按下两个键。

\begin{ccode}
  while (1)
  {
    if (!button)
    {
      for (; button != 3; --time);
      time = COUNT;
    }
    else
    {
      switch (button)
      {
      case 1:
        // turn off the other lights
        MAP_GPIO_setOutputLowOnPin(GPIO_PORT_P2, GPIO_PIN0);
        MAP_GPIO_setOutputLowOnPin(GPIO_PORT_P2, GPIO_PIN1);

        if (!MAP_FlashCtl_performMassErase()) // failed to erase
        {
          // turn off LED1 R, turn on LED2 B
          MAP_GPIO_setOutputHighOnPin(GPIO_PORT_P1, GPIO_PIN0);
          MAP_GPIO_setOutputLowOnPin(GPIO_PORT_P2, GPIO_PIN2);
        }
\end{ccode}

\begin{ccode}
        else // succeed to erase
        {
          // opposite
          MAP_GPIO_setOutputLowOnPin(GPIO_PORT_P1, GPIO_PIN0);
          MAP_GPIO_setOutputHighOnPin(GPIO_PORT_P2, GPIO_PIN2);
        }
        break;
      
      case 2:
        // turn off the other lights
        MAP_GPIO_setOutputLowOnPin(GPIO_PORT_P2, GPIO_PIN0);
        MAP_GPIO_setOutputLowOnPin(GPIO_PORT_P2, GPIO_PIN2);
        if (!MAP_FlashCtl_programMemory(array, (void *)BANKl_S29, SIZE))
        {
          // turn off LED2 G, turn on LED1 R
          MAP_GPIO_setOutputHighOnPin(GPIO_PORT_P1, GPIO_PIN0);
          MAP_GPIO_setOutputLowOnPin(GPIO_PORT_P2, GPIO_PIN1);
        }

        else
        {
          // opposite
          MAP_GPIO_setOutputLowOnPin(GPIO_PORT_P1, GPIO_PIN0);
          MAP_GPIO_setOutputHighOnPin(GPIO_PORT_P2, GPIO_PIN1);
        }
        break;
      case 3:

        // randomly select a position in BANK1_S29
        pos = rand() % SIZE;
        S29 = *((unsigned char *)(BANKl_S29 + pos));

        // if the char is what we want
        if (S29 == strNum)
        {
          // turn on red, green, blue light on LED2 only,
          // which is white light
          MAP_GPIO_setOutputHighOnPin(GPIO_PORT_P2, GPIO_PIN0);
          MAP_GPIO_setOutputHighOnPin(GPIO_PORT_P2, GPIO_PIN1);
          MAP_GPIO_setOutputHighOnPin(GPIO_PORT_P2, GPIO_PIN2);
          MAP_GPIO_setOutputLowOnPin(GPIO_PORT_P1, GPIO_PIN0);
        }
        else
        {
          // only red light on LED1 only
          MAP_GPIO_setOutputLowOnPin(GPIO_PORT_P2, GPIO_PIN0);
          MAP_GPIO_setOutputLowOnPin(GPIO_PORT_P2, GPIO_PIN1);
          MAP_GPIO_setOutputLowOnPin(GPIO_PORT_P2, GPIO_PIN2);
          MAP_GPIO_setOutputHighOnPin(GPIO_PORT_P1, GPIO_PIN0);
        }

        break;
      default:
        break;
      }
      button = 0;
    }
  }
\end{ccode}

主循环最开始用来获取信号,等待两个键同时按下(通过跑空循环的方式)。然后就是分别对三种情况按要求操作。使用之前用到的 \mintinline{c}|MAP_FlashCtl_performMassErase| 进行整体擦除,使用
\\ \mintinline{c}|MAP_GPIO_setOutputLowOnPin| 关闭灯,使用 \mintinline{c}|MAP_GPIO_setOutputHighOnPin| 开启灯,\\ 
使用 \mintinline{c}|MAP_FlashCtl_programMemory| 写入数据,使用指针取地址直接读取数据。这里是随机读入数据。

课堂上和助教检查结果,符合 ppt 要求
\section{总结}

https://www.ti.com/tool/MSP-EXP432P401R\#technicaldocuments 给予了有用的介绍和思路

\section{附件}

\end{document}
