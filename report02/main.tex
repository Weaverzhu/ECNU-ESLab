% !TEX encoding = UTF-8 Unicode
\RequirePackage{fix-cm}
\documentclass[a4paper,10pt,UTF8]{paper}
%\documentclass[a4paper,10pt,UTF8]{ctexart}

\usepackage[english]{babel}
\usepackage{fancyhdr,array,lastpage,amsmath,mathtools,enumitem,graphicx,multirow,tocbibind,longtable,makecell,varwidth,titlesec,bm,booktabs,comment}
\usepackage{enumitem}
\usepackage{hyperref}
\hypersetup{hidelinks}
%\setCJKmainfont[BoldFont=Heiti SC Medium]{Songti SC Light}
%\setCJKsansfont{Heiti SC}

\usepackage[left=2.54cm,right=2.54cm,top=2.54cm,bottom=2.54cm]{geometry}
\usepackage[font=footnotesize,labelfont=bf]{caption}
\usepackage{tikz,flowchart}
\usepackage{ctex}
\usetikzlibrary{shapes,shapes.geometric,arrows,matrix,calc}
\usetikzlibrary{circuits.logic}
% \usetikzlibrary{circuits.logic.custom}
\usetikzlibrary{circuits.logic.IEC}
\usetikzlibrary{shadows}
\usepackage{listings}
\usepackage[Q=yes]{examplep}
\usepackage{fancyhdr}
\usepackage{alphalph}
\usepackage{indentfirst}

\newenvironment{sol}
  {\par\vspace{2mm}\noindent{\bf Solution}. }

\lstset{escapeinside=``, breaklines=true, frame=none, extendedchars=false, basicstyle=\ttfamily, showstringspaces=false}


\setlength{\parindent}{2em}
\setlength{\parskip}{1.5ex plus 0.5ex minus 0.2ex}
\linespread{1.1}

\bibliographystyle{plain}

\numberwithin{equation}{section}
\numberwithin{figure}{section}

\usepackage{karnaugh}
\usepackage{circuitikz}


\setcounter{secnumdepth}{3}
\setcounter{tocdepth}{3}

\title{华东师范大学计算机科学技术系上机实验报告}

\begin{document}
\pagestyle{fancy}
\chead{\small\color{gray}华东师范大学计算机科学技术系上机实验报告}
\lhead{}
\rhead{}
\makeatletter
\def\headrule{{\if@fancyplain\let\headrulewidth\plainheadrulewidth\fi%
\color{gray}\hrule\@height 0.2pt\@width\headwidth}
  \vspace{6mm}}
\makeatother

\newcommand{\HRule}{\rule{\linewidth}{1mm}}
\newcommand{\dai}{\textbf{Dais-CMX16$^+$}}

{\center {\huge \bfseries \LARGE{华东师范大学计算机科学技术系上机实验报告}} \\ [0.8cm]

\small{
  \begin{minipage}[t]{.32\linewidth}
    \textbf{课程名称:} 嵌入式系统\\
    \textbf{指导教师:} 沈建华\\
    \textbf{上机实践名称:} 汇编语言程序设计\\
    \textbf{实践编号:}实验2
  \end{minipage}
  \begin{minipage}[t]{.32\linewidth}
    \textbf{年级:}17 级\\
    \textbf{姓名:}朱桐\\
    \textbf{学号:}10175102111\\
  \end{minipage} 
  \begin{minipage}[t]{.32\linewidth}
    \textbf{上机实践成绩:} \\
    \textbf{创新实践成绩:} \\
    \textbf{上机实践日期:}2019/10/08\\
    \textbf{上机实践时间:}2 学时\\
  \end{minipage}
}
\HRule \\[0.5cm]
}
\section{实验目的}

\begin{enumerate}
    \item 学习并掌握简单的ARM汇编指令(MOV、ADD、SUB、CMP、LDR、逻辑指令等),熟悉其功能及用法
    \item 熟悉基本汇编程序的程序语法及设计格式,能写出简单的汇编程序
    \item 学会调试汇编程序,能及时发现并改正错误
\end{enumerate}

\section{实验设备}

软件开发系统Keil5(Keil提供了软件仿真功能)

\section{实验内容}

\begin{enumerate}
  \item 学习Keil中的汇编语言格式
  \item 对例子程序进行单步调试,观察寄存器和内存单元的值变化
\end{enumerate}

\section{实验原理}

\subsection{ARM汇编语言格式}

在ARM汇编程序中,所有的标号必须顶格写,所有的指令均不能顶格书写,指令前面应该有空格,一般用Tab健。

因为ARM汇编器对标志符的大小写敏感,因此书写标志及指令时,大小写要一致。在ARM汇编程序中,指令、寄存器名可以全部为大写,也可以全部为小写,但是不能大小写混合使用;

在ARM汇编程序中使用注释,注释内容由“;”开始一直到此行结束,注释可以顶格写;
注意:
	“;”相当于C语言中的“//”

定义变量、常量时,其标识符必须顶格书写,否则编译器报错

函数名需要顶格写。



\section{实验步骤}

\subsection{练习1}

使用正常的方法,我们需要一个分支语句

\section{调试过程、结果与分析}

\section{总结}

\section{附件}

\end{document}
