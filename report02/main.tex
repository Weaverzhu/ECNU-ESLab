% !TEX encoding = UTF-8 Unicode
\RequirePackage{fix-cm}
\documentclass[a4paper,10pt,UTF8]{paper}
%\documentclass[a4paper,10pt,UTF8]{ctexart}

\usepackage[english]{babel}
\usepackage{fancyhdr,array,lastpage,amsmath,mathtools,enumitem,graphicx,multirow,tocbibind,longtable,makecell,varwidth,titlesec,bm,booktabs,comment,listings,minted}
\usepackage{enumitem}
\usepackage{hyperref}
\hypersetup{hidelinks}
% \setCJKmainfont[BoldFont=PingFang SC]{PingFang SC}
% \setCJKsansfont{PingFang SC}

\usepackage[left=2.54cm,right=2.54cm,top=2.54cm,bottom=2.54cm]{geometry}
\usepackage[font=footnotesize,labelfont=bf]{caption}
\usepackage{tikz,flowchart}
\usepackage{ctex}
\usetikzlibrary{shapes,shapes.geometric,arrows,matrix,calc}
\usetikzlibrary{circuits.logic}
% \usetikzlibrary{circuits.logic.custom}
\usetikzlibrary{circuits.logic.IEC}
\usetikzlibrary{shadows}
\usepackage{listings}
\usepackage[Q=yes]{examplep}
\usepackage{fancyhdr}
\usepackage{alphalph}
\usepackage{indentfirst}

\hypersetup{
colorlinks=true,
linkcolor=black
}

\newenvironment{sol}
  {\par\vspace{2mm}\noindent{\bf Solution}. }

\lstset{escapeinside=``, breaklines=true, frame=none, extendedchars=false, basicstyle=\ttfamily, showstringspaces=false}


\setlength{\parindent}{2em}
\setlength{\parskip}{1.5ex plus 0.5ex minus 0.2ex}
\linespread{1.1}

\bibliographystyle{plain}

\numberwithin{equation}{section}
\numberwithin{figure}{section}

\lstset{
 columns=fixed,       
 numbers=left,                                        % 在左侧显示行号
 numberstyle=\tiny\color{gray},                       % 设定行号格式
 frame=none,                                          % 不显示背景边框
 backgroundcolor=\color[RGB]{245,245,244},            % 设定背景颜色
 keywordstyle=\color[RGB]{40,40,255},                 % 设定关键字颜色
 numberstyle=\footnotesize\color{darkgray},           
 commentstyle=\it\color[RGB]{0,96,96},                % 设置代码注释的格式
 stringstyle=\rmfamily\slshape\color[RGB]{128,0,0},   % 设置字符串格式
 showstringspaces=false,                              % 不显示字符串中的空格
language=c++,
}

\usepackage{karnaugh}
\usepackage{circuitikz}


\setcounter{secnumdepth}{3}
\setcounter{tocdepth}{3}

\title{华东师范大学计算机科学技术系上机实验报告}

\begin{document}
\pagestyle{fancy}
\chead{\small\color{gray}华东师范大学计算机科学技术系上机实验报告}
\lhead{}
\rhead{}
\makeatletter
\def\headrule{{\if@fancyplain\let\headrulewidth\plainheadrulewidth\fi%
	\color{gray}\hrule\@height 0.2pt\@width\headwidth}
	\vspace{6mm}}
\makeatother

\newcommand{\HRule}{\rule{\linewidth}{1mm}}
\newcommand{\dai}{\textbf{Dais-CMX16$^+$}}

{\center {\huge \bfseries \LARGE{华东师范大学计算机科学技术系上机实验报告}} \\ [0.8cm]
		
	\small{
		\begin{minipage}[t]{.32\linewidth}
			\textbf{课程名称:} 嵌入式系统\\
			\textbf{指导教师:} 沈建华\\
			\textbf{上机实践名称:} 汇编语言程序设计\\
			\textbf{实践编号:}实验2
		\end{minipage}
		\begin{minipage}[t]{.32\linewidth}
			\textbf{年级:}17 级\\
			\textbf{姓名:}朱桐\\
			\textbf{学号:}10175102111\\
		\end{minipage} 
		\begin{minipage}[t]{.32\linewidth}
			\textbf{上机实践成绩:} \\
			\textbf{创新实践成绩:} \\
			\textbf{上机实践日期:}2019/10/08\\
			\textbf{上机实践时间:}2 学时\\
		\end{minipage}
	}
	\HRule \\[0.5cm]
}
\section{实验目的}

\begin{enumerate}
	\item 学习并掌握简单的ARM汇编指令(MOV、ADD、SUB、CMP、LDR、逻辑指令等),熟悉其功能及用法
	\item 熟悉基本汇编程序的程序语法及设计格式,能写出简单的汇编程序
	\item 学会调试汇编程序,能及时发现并改正错误
\end{enumerate}

\section{实验设备}

软件开发系统Keil5(Keil提供了软件仿真功能)

\section{实验内容}

\begin{enumerate}
	\item 学习Keil中的汇编语言格式
	\item 对例子程序进行单步调试,观察寄存器和内存单元的值变化
\end{enumerate}

\section{实验原理}

\subsection{ARM汇编语言格式}

在ARM汇编程序中,所有的标号必须顶格写,所有的指令均不能顶格书写,指令前面应该有空格,一般用Tab健。

因为ARM汇编器对标志符的大小写敏感,因此书写标志及指令时,大小写要一致。在ARM汇编程序中,指令、寄存器名可以全部为大写,也可以全部为小写,但是不能大小写混合使用;

在ARM汇编程序中使用注释,注释内容由“;”开始一直到此行结束,注释可以顶格写;
注意:
“;”相当于C语言中的“//”

定义变量、常量时,其标识符必须顶格书写,否则编译器报错

函数名需要顶格写。


\subsection{while 循环}

我们已经有循环的概念,那么怎么在汇编中使用循环呢?

\begin{minted}[mathescape,
    linenos,
    numbersep=5pt,
    gobble=2,
    frame=lines,
    framesep=2mm]{asm}

    MOV R1,#10
    loop
        SUBS R1,R1,#1
        BNE loop

\end{minted}

我们使用分支跳转语句,可以做一个类似while 循环,上面代码相当于用 r1 寄存器作为循环变量,做while循环,等价于

\begin{minted}[mathescape,
    linenos,
    numbersep=5pt,
    gobble=2,
    frame=lines,
    framesep=2mm]{c}

    r1 = 10;
    do {
        r1 = r1 - 1;
    } while (r1 >= 0);

\end{minted}


\subsection{if 语句}

我们可以使用一个标记来实现一个没有 elses 的if语句

\begin{minted}[mathescape,
    linenos,
    numbersep=5pt,
    gobble=2,
    frame=lines,
    framesep=2mm]{asm}

    SUBS R2,R1,#1
    BNE lab;
        ; do something
    lab

\end{minted}

上述代码相当于 

\begin{minted}[mathescape,
    linenos,
    numbersep=5pt,
    gobble=2,
    frame=lines,
    framesep=2mm]{c}

    if ((r2 = r1-1) == 0) {
        // do something
    }

\end{minted}

BNE, BEQ, BLE 等分支跳转语句将会自动获取上一次运算命令的运算结果,然后根据情况进行跳转。

\section{实验步骤}

\subsection{练习1}

使用正常的方法,我们需要一个分支语句,也就是类似c语言中的 if 语句,而没有else即 "if ((x\&1) == 1) sum += x"。于是我们可以使用一个标号完成这个过程。将循环跳转语句即为 cont,循环体内做的操作记为 even,如果不满足if则直接跳转至 conti,就能完成这种if语句。

我们只需要根据 1加到10的程序稍加修改就能得到结果

\subsection{练习2}

我们同样需要一个 if 没有 else 的语句。即 "if (x > 10) break。"

然后就是怎么求斐波那契数列。已知前两项我们就能求出后一项,我们可以用汇编翻译下列c语言代码。

\begin{minted}[mathescape,
    linenos,
    numbersep=5pt,
    gobble=2,
    frame=lines,
    framesep=2mm]{c}
    r1 = 1;
    r2 = 1;
    r0 = 0;
    do {
        r0 = r0 + r3;
        r3 = r1 + r2;
        r1 = r2;
        r2 = r3;
        if (r3 - 10 > 10) break;
    } while (1);

\end{minted}

其中 if , break 可以修改为如果小于等于10就继续循环,可以不添加新的标识。


\begin{minted}[mathescape,
    linenos,
    numbersep=5pt,
    gobble=2,
    frame=lines,
        framesep=2mm]{asm}
        MOV R1,#0
        MOV R2,#1
        MOV R0,#0
        MOV R3,#0
        ADD R3,R1,R2
    start
        ADD R0,R0,R3
        ADD R3,R1,R2
        MOV R1,R2
        MOV R2,R3
        SUBS R4,R3,#10
        BLE start
        NOP
        ENDP
        END

\end{minted}

\section{调试过程、结果与分析}

\subsection{练习1}

结果为 1E,即30,而 2+4+6+8+10 = 30,结果正确

\subsection{练习2}

结果为 14,而 1+1+2+3+5+8 = 20,结果正确

\section{总结}

这节实验体验了汇编语言的编写,用 keil5 软件编写十分便捷,调试方便。

总结了一下经验

\begin{itemize}
    \item 我们可以先思考 c 语言怎么写,然后翻译过来
    \item 基础的赋值语句可以使用 MOV
    \item if语句可以使用添加标记和分支跳转
    \item while语句可以结合分支跳转或者无条件跳转
    \item 熟练使用单步调试可以在以后写更复杂程序时提供标识
    \item 思考使用更少的语句和标识可以提高程序的运行速度
\end{itemize}

\section{附件}

练习1代码
\begin{minted}[mathescape,
    linenos,
    numbersep=5pt,
    gobble=2,
    frame=lines,
    framesep=2mm]{asm}
    AREA Reset,DATA,READONLY;
    DCD 0X12345678
    DCD   Reset_Handler
        AREA CODE_SEGMET,CODE,READONLY;
        ENTRY
    Reset_Handler  proc
        export Reset_Handler    [weak]
        MOV R0,#0;?????
        MOV R1,#10;
    start
        ANDS R2,R1,#1
        BNE cont
        ADDS R0,R0,R1    
    cont
        SUBS R1,R1,#1

        BNE start
        NOP
        ENDP
        END

\end{minted}

练习2代码

\begin{minted}[mathescape,
    linenos,
    numbersep=5pt,
    gobble=2,
    frame=lines,
        framesep=2mm]{asm}
        AREA Reset,DATA,READONLY;
        DCD 0X12345678
        DCD   Reset_Handler
        AREA CODE_SEGMET,CODE,READONLY;
        ENTRY
    Reset_Handler  proc
        export Reset_Handler    [weak]
        MOV R1,#0
        MOV R2,#1
        MOV R0,#0
        MOV R3,#0
        ADD R3,R1,R2
    start
        ADD R0,R0,R3
        ADD R3,R1,R2
        MOV R1,R2
        MOV R2,R3
        SUBS R4,R3,#10
        BLE start
        NOP
        ENDP
        END

\end{minted}


\end{document}
