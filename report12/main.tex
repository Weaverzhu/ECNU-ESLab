% !TEX encoding = UTF-8 Unicode
\RequirePackage{fix-cm}
\documentclass[a4paper,10pt,UTF8]{paper}
%\documentclass[a4paper,10pt,UTF8]{ctexart}

\usepackage[english]{babel}
\usepackage{fancyhdr,array,lastpage,amsmath,mathtools,enumitem,graphicx,multirow,tocbibind,longtable,makecell,varwidth,titlesec,bm,booktabs,comment,minted}
\usepackage{enumitem}
\usepackage{hyperref}
\hypersetup{hidelinks}




\usepackage[left=2.54cm,right=2.54cm,top=2.54cm,bottom=2.54cm]{geometry}
\usepackage[font=footnotesize,labelfont=bf]{caption}
\usepackage{tikz,flowchart}
\usepackage{ctex}
\usepackage{xeCJK}%中文字体
\usetikzlibrary{shapes,shapes.geometric,arrows,matrix,calc}
\usetikzlibrary{circuits.logic}

% \usetikzlibrary{circuits.logic.custom}
\usetikzlibrary{circuits.logic.IEC,automata,positioning}
\usetikzlibrary{shadows}
\usepackage{listings}
\usepackage[Q=yes]{examplep}
\usepackage{fancyhdr}
\usepackage{alphalph}
\usepackage{indentfirst}

% \setCJKsansfont{黑体}
\setmainfont{PingFang SC}
\setCJKmainfont{PingFang SC}
\setCJKsansfont{PingFang SC}
\setmonofont{Monaco}

\newenvironment{sol}
  {\par\vspace{2mm}\noindent{\bf Solution}. }

\lstset{escapeinside=``, breaklines=true, frame=none, extendedchars=false, basicstyle=\ttfamily, showstringspaces=false}


\setlength{\parindent}{2em}
\setlength{\parskip}{1.5ex plus 0.5ex minus 0.2ex}
\linespread{1.1}

\bibliographystyle{plain}

\numberwithin{equation}{section}
\numberwithin{figure}{section}


\setcounter{secnumdepth}{3}
\setcounter{tocdepth}{3}

\title{华东师范大学计算机科学技术系上机实验报告}

\begin{document}
\pagestyle{fancy}
\chead{\small\color{gray}华东师范大学计算机科学技术系上机实验报告}
\lhead{}
\rhead{}
\makeatletter
\def\headrule{{\if@fancyplain\let\headrulewidth\plainheadrulewidth\fi%
    \color{gray}\hrule\@height 0.2pt\@width\headwidth}
    \vspace{6mm}}
\makeatother

\newcommand{\HRule}{\rule{\linewidth}{1mm}}
\newcommand{\dai}{\textbf{Dais-CMX16$^+$}}

{\center {\huge \bfseries \LARGE{华东师范大学计算机科学技术系上机实验报告}} \\ [0.8cm]
                            
    \small{
        \begin{minipage}[t]{.32\linewidth}
            \textbf{课程名称:}嵌入式原理与实践\\
            \textbf{指导教师:}沈建华\\
            \textbf{上机实践名称:} 综合练习\\
            \textbf{实践编号:}实验 12
        \end{minipage}
        \begin{minipage}[t]{.32\linewidth}
            \textbf{年级:}17 级\\
            \textbf{姓名:}朱桐\\
            \textbf{学号:}10175102111\\
            \textbf{组号:}A
        \end{minipage} 
        \begin{minipage}[t]{.32\linewidth}
            \textbf{上机实践成绩:} \\
            \textbf{创新实践成绩:} \\
            \textbf{上机实践日期:}2019/12/24\\
            \textbf{上机实践时间:}2 学时\\
        \end{minipage}
    }
    \HRule \\[0.5cm]
}

\definecolor{bg}{rgb}{0.95,0.95,0.95}
\newminted{asm}{bgcolor=bg}
\newminted{c}{bgcolor=bg}

\NewDocumentCommand{\cw}{v}{\texttt{\textcolor{black}{#1}}}

\section{实验目的}

\begin{itemize}
    \item 综合运用之前几次实验的内容
\end{itemize}

\section{实验设备}

\begin{itemize}
    \item MSP-EXP432P401R LaunchPad开发板
\end{itemize}

\section{实验内容}

\section{实验步骤}

为了完成一个比较复杂的功能,我们不应该心急一次想好所有的内容,而是吸取模块化思想,一次完成一个相对较小的模块,才能快速有效地设计好整个思路

这里将整个功能分成三个模块

\begin{itemize}
    \item URAT 端口与串口工具的交互
    \item Timer32 控制 LED2 亮度改变,周期为 1s
    \item TimerA 控制 LED2 的亮度大小,要求0到9由暗到亮
\end{itemize}

\subsection{UART 串口}

首先考虑UART 串口进行的操作

\begin{itemize}
    \item 读入数字直到空格,此时为读入状态
    \item 在输出的时候如果执行读入,则打断继续到读入状态,或者忽略读入继续输出,这里采用第二种方案
    \item 可以使用自适应长度(读到空格位置),或者限定只能读入三位。这里采用第一种方案
\end{itemize}

整理状态,有如下有限状态自动机。虽然画图可以非常思路清晰地展现程序地逻辑,但是我们之前已经将功能分割成小的模块,一般在写程序的时候不会画图

对于URAT 控制我们一直输入字符直到空格位置,然后切换输入状态

% \begin{figure}[h]
%     \centering
%     \begin{tikzpicture}[->,>=stealth',shorten >=1pt,auto,node distance=2cm,
%         thick,base node/.style={circle,draw,minimum size=pt60}, real node/.style={double,circle,draw,minimum size=60pt}]
    
    
    
%         \node[initial,initial text={}, accepting, state] (1) {$\epsilon$};
%         \node[state](2)[right of=1 ]{$51$}[above];
        
%         \path[]
%         (1) edge [loop below]node {$0,101$} (1)
%         (1) edge node {$51$} (2)
%         (2) edge [loop above]node {$0$} (2);
        
%     \end{tikzpicture}
%     \caption{URAT 串口的状态自动机}
%     \label{fig:urat_dfa}
% \end{figure}


\begin{figure}[h]
    \centering
    \begin{tikzpicture}[->,>=stealth',shorten >=1pt,auto,node distance=4cm,
        thick,base node/.style={circle,draw,minimum size=60pt}, real node/.style={double,circle,draw,minimum size=60pt}]
                                                    
        \node[initial, initial text={}, state,minimum size = 2cm] (1) {等待输入};
        \node[state,minimum size = 2cm] (2) [right of = 1] {输出第一位};
        \node[state,minimum size = 2cm] (3) [below of = 2] {输出第二位};
        \node[minimum size = 2cm] (5) [below of = 3] {...};
        \node[state,minimum size = 2cm] (4) [below of = 5] {最后一位};
                                                
                                                
        \path[]
        (1) edge [loop above] node {输入数字} (1)
        (1) edge node {输入空格} (2)
        (2) edge node {输出,保留1s,指针切换下一位} (3)
        (3) edge node {输出,保留1s,指针切换下一位} (5)
        (5) edge node {输出,保留1s,指针切换下一位} (4)
        (4) edge node {输出,保留,初始化指针} (1);
                                                        
                                                
                                                        
    \end{tikzpicture}
    \caption{URAT 串口的状态自动机}
    \label{fig:urat_dfa}
\end{figure}

\subsection{Timer32}

Timer32 用于控制 1s 的周期,在这部分中我们只考虑在输出的状态中,因为不输出的时候时钟是关掉的。状态自动机如图 \ref{fig:timera_dfa}

Timer32 每次中断的时候代表 1s 周期到了然后切换下一个字符


\begin{figure}[h]
    \centering
    \begin{tikzpicture}[->,>=stealth',shorten >=1pt,auto,node distance=4cm,
        thick,base node/.style={circle,draw,minimum size=60pt}, real node/.style={double,circle,draw,minimum size=60pt}]
                                                    
        \node[initial, initial text={}, state,minimum size = 2cm] (1) {等待输入};
        \node[state,minimum size = 2cm] (2) [right of = 1] {正在输出};
                                                
                                                
        \path[]
        (1) edge [loop above] node {输入数字} (1)
        (1) edge [bend left] node {输入空格} (2)
        (2) edge [loop above] node {中断,输出下一个字符}
        (2) edge[bend left] node {输出最后一个数字,关闭时钟} (1);
                                                        
                                                                
    \end{tikzpicture}
    \caption{Timer32时钟状态自动机}
    \label{fig:timer32_dfa}
\end{figure}

TimerA 也是随着 Timer32 中断,改变自身的占空比为下一个字符。如图 \ref{fig:timera_dfa}

\begin{figure}[h]
    \centering
    \begin{tikzpicture}[->,>=stealth',shorten >=1pt,auto,node distance=4cm,
        thick,base node/.style={circle,draw,minimum size=60pt}, real node/.style={double,circle,draw,minimum size=60pt}]
                                                    
        \node[initial, initial text={}, state,minimum size = 2cm] (1) {等待输入};
        \node[state,minimum size = 2cm] (2) [right of = 1] {正在输出};
                                                
                                                
        \path[]
        (1) edge [loop above] node {输入数字} (1)
        (1) edge [bend left] node {输入空格} (2)
        (2) edge [loop above] node {中断,占空比改变}
        (2) edge[bend left] node {输出最后一个数字,关闭时钟,保持占空比} (1);
                                                        
                                                                
    \end{tikzpicture}
    \caption{TimerA 时钟状态自动机}
    \label{fig:timera_dfa}
\end{figure}

\section{代码分析}

配置单通道 LED2 蓝灯,GPIO 2.2, Timer32,TimerA,URAT 端口

\begin{ccode}
                    
        
    const Timer_A_UpModeConfig TA0 =
    {
        TIMER_A_CLOCKSOURCE_SMCLK,  
        TIMER_A_CLOCKSOURCE_DIVIDER_1,  
        PWM_PERIODS - 1,  
        TIMER_A_TAIE_INTERRUPT_DISABLE,  
        TIMER_A_CCIE_CCR0_INTERRUPT_DISABLE,
        TIMER_A_DO_CLEAR  
    };
    
    
    Timer_A_CompareModeConfig TA0_CCR3_PWM =
    {
        TIMER_A_CAPTURECOMPARE_REGISTER_3,   
        TIMER_A_CAPTURECOMPARE_INTERRUPT_DISABLE,  
        TIMER_A_OUTPUTMODE_RESET_SET,  
    PWM_PERIODS};
    
    const eUSCI_UART_Config uartConfig =
    {
        EUSCI_A_UART_CLOCKSOURCE_SMCLK,
        6,
        8,
        17,
        EUSCI_A_UART_NO_PARITY,
        EUSCI_A_UART_LSB_FIRST,
        EUSCI_A_UART_ONE_STOP_BIT,
        EUSCI_A_UART_MODE, 
    EUSCI_A_UART_OVERSAMPLING_BAUDRATE_GENERATION};

\end{ccode}
    
\begin{ccode}
    uint8_t port_mapping[] =
    {
        PM_NONE,
        PM_NONE,
        PM_TA0CCR3A,
        PM_NONE,
        PM_NONE,
        PM_NONE,
        PM_NONE,
    PM_NONE};
                        
    MAP_WDT_A_holdTimer();
                        
    //![Simple FPU Config]
    MAP_FPU_enableModule();
    MAP_CS_setDCOFrequency(DCO_Frequency);
    MAP_CS_initClockSignal(CS_MCLK, CS_DCOCLK_SELECT, CS_CLOCK_DIVIDER_1);
    MAP_CS_initClockSignal(CS_HSMCLK, CS_DCOCLK_SELECT, CS_CLOCK_DIVIDER_4);
    MAP_CS_initClockSignal(CS_SMCLK, CS_DCOCLK_SELECT, CS_CLOCK_DIVIDER_4);
    //![Simple FPU Config]
                        
    MAP_GPIO_setAsOutputPin(GPIO_PORT_P2, GPIO_PIN0);
    MAP_GPIO_setOutputLowOnPin(GPIO_PORT_P2, GPIO_PIN0);
    MAP_GPIO_setAsOutputPin(GPIO_PORT_P2, GPIO_PIN1);
    MAP_GPIO_setOutputLowOnPin(GPIO_PORT_P2, GPIO_PIN1);
    MAP_GPIO_setAsOutputPin(GPIO_PORT_P2, GPIO_PIN2);
    MAP_GPIO_setOutputLowOnPin(GPIO_PORT_P2, GPIO_PIN2);
                        
    MAP_Timer32_initModule(TIMER32_0_BASE, TIMER32_PRESCALER_1, 
    TIMER32_32BIT, TIMER32_PERIODIC_MODE);
    MAP_Interrupt_enableInterrupt(INT_T32_INT1);
    MAP_Timer32_enableInterrupt(TIMER32_0_BASE);
    MAP_Timer32_clearInterruptFlag(TIMER32_0_BASE);
                        
    MAP_GPIO_setAsPeripheralModuleFunctionInputPin(GPIO_PORT_P1, 
    GPIO_PIN2 | GPIO_PIN3, GPIO_PRIMARY_MODULE_FUNCTION);
    MAP_UART_initModule(EUSCI_A0_BASE, &uartConfig);
    MAP_UART_enableModule(EUSCI_A0_BASE);
    MAP_UART_enableInterrupt(EUSCI_A0_BASE, EUSCI_A_UART_RECEIVE_INTERRUPT);
    MAP_Interrupt_enableInterrupt(INT_EUSCIA0);
                        
    MAP_PMAP_configurePorts((const uint8_t *)port_mapping, PMAP_P2MAP, 
    1 , PMAP_DISABLE_RECONFIGURATION);
    MAP_GPIO_setAsPeripheralModuleFunctionOutputPin(GPIO_PORT_P2, 
    GPIO_PIN2, GPIO_PRIMARY_MODULE_FUNCTION);
    MAP_Timer_A_configureUpMode(TIMER_A0_BASE, &TA0);
    MAP_Timer_A_startCounter(TIMER_A0_BASE, TIMER_A_UP_MODE);
                        
    MAP_Interrupt_enableInterrupt(INT_TA1_0);
    TA0_CCR3_PWM.compareValue = 0;
    output = 0;
\end{ccode}

URAT 中断函数,如果在输出,阻塞输入,否则增加输出。如果输入数字,保存在 \texttt{buf} 中。如果是空格,则开启输出模式

\begin{ccode}
    void EUSCIA0_IRQHandler(void)
    {
        char tmp[100];
                        
        uint32_t status = MAP_UART_getEnabledInterruptStatus(EUSCI_A0_BASE);
        MAP_UART_clearInterruptFlag(EUSCI_A0_BASE, status);
        if (output)
        {
            output = 0;
            cur = 0;
            TA0_CCR3_PWM.compareValue = 0;
            start = 0;
            MAP_Timer_A_stopTimer(TIMER_A0_BASE);
            MAP_Timer32_haltTimer(TIMER32_0_BASE);
        }
        if (status & EUSCI_A_UART_RECEIVE_INTERRUPT_FLAG)
        {
            char ch = MAP_UART_receiveData(EUSCI_A0_BASE);
            if (ch == ' ')
            {
                len = cur;
                cur = 0;
                output = 1;
                                                
                MAP_Timer32_setCount(TIMER32_0_BASE, Duration * DCO_Frequency);
                MAP_Timer32_startTimer(TIMER32_0_BASE, false);
                                                
                sprintf(tmp, "start to timer\n");
                for (int i = 0; i < strlen(tmp); ++i)
                {
                    MAP_UART_transmitData(EUSCI_A0_BASE, tmp[i]);
                }
            }
            else
            {
                buf[cur++] = ch;
            }
        }
    }
\end{ccode}

TimerA 中断函数,如果输出到末尾会返回等待输入模式,每次中断会改变输出的字符。输出的方式就是控制 PWM 的占空比

\begin{ccode}
    void T32_INT1_IRQHandler(void)
    {
        char tmp[100];
        if (output == 0) return;
        MAP_Timer32_clearInterruptFlag(TIMER32_0_BASE);
        char ch = buf[cur++];
        sprintf(tmp, "ch=%c\n", ch);
        for (int i = 0; i < strlen(tmp); ++i)
        {
            MAP_UART_transmitData(EUSCI_A0_BASE, tmp[i]);
        }
        if (cur == len)
        {
            output = 0;
            cur = 0;
            start = 0;
        }
        int digit = ch - '0' + 1;
        // if (digit == 0)
        //   digit = 10;
        double coff = 1.0 * digit / 10;
        TA0_CCR3_PWM.compareValue = PWM_PERIODS * coff;
        MAP_Timer_A_initCompare(TIMER_A0_BASE, &TA0_CCR3_PWM);
                
        MAP_Timer32_setCount(TIMER32_0_BASE, Duration * DCO_Frequency);
        MAP_Timer32_startTimer(TIMER32_0_BASE, false);
    }
\end{ccode}
\section{总结}

该设计还有额外特性:

\begin{itemize}
    \item 只要没有超过 \texttt{buf} 数组大小,可以接受任意长度的数字
    \item 可以通过 URAT 串口进行调试
    \item 注意隐式类型转换
\end{itemize}
    

\section{附件}

\end{document}
