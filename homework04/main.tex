\documentclass{article}
\usepackage[utf8]{inputenc}
\usepackage{indentfirst, tabularx,geometry}
\geometry{left=3.0cm,right=3.0cm,top=3.5cm,bottom=3.5cm}
\usepackage{ctex}
\usepackage{xeCJK}%中文字体

\setmainfont{PingFang SC}
\setCJKmainfont{PingFang SC}
\setCJKsansfont{PingFang SC}


\title{第四次课后作业}
\author{10175102111 朱桐}
\date{October 2019}

\begin{document}


\maketitle


\setcounter{section}{4}
% \section{第四章}

\setcounter{subsection}{0}
\subsection{MCU的复位有什么作用?为何需要有多个复位源?}

在嵌入式系统电子设备的运行中,当出现程序跑飞的情况或程序跳转时,可用手动或自动的方法发信号给硬件特定接口,使软件恢复到特定的程序段运行。按照复位源的不同可以将复位分为系统复位、电源复位。

不管是系统复位还是电源复位都会将全部寄存器复位为复位值,所有的这些复位源都被引入复位控制器,复位控制器根据不同的复位源产生不同的复位动作。

\setcounter{subsection}{2}
\subsection{为什么 MCU 要支持多种时钟源?查阅资料说明 MCU 内部低速 RC 时钟、内部高速 RC 时钟、外部低速时钟、外部高速时钟的特点和适用性。}

当前嵌入式硬件的时钟系统比较复杂,不再能够用一个系统时钟满足所有需求。

硬件板上电路的时钟越快,功耗越大,且其抗电磁干扰能力也会减弱,所以不可能一味地提高时钟频率来满足各种设备的需求,因此较为复杂的MCU一般都采用多时钟源的方法。特点如图 \ref{fig:1} 所示

\begin{figure}[h]
  \centering
  \begin{tabularx}{0.9\textwidth}{|X|X|X|}
    \hline & 特点 & 适用性 \\ \hline
    低速内部时钟(LSI) &	片内 RC 电阻、电容时钟振荡器产生 & 实时时钟模块、看门狗模块 \\ \hline 
    高速内部时钟(HSI) &	片内 RC 电阻、电容时钟振荡器产生 & 初始系统时钟 \\ \hline 
    低速外部时钟(LSE) & 外部晶振作为时钟源,晶体一般选择一个低速外部晶体或陶瓷谐振器 &为实时时钟或者其他定时功能提供一个低功耗且精确的时钟源 \\ \hline
    高速外部时钟(HSE)& 外部晶体作为时钟源,常用的晶体频率根据芯片的不同而发生变化& 系统时钟 \\ \hline
  \end{tabularx}
  \caption{4.3}
  \label{fig:1}
\end{figure}

\setcounter{subsection}{3}
\subsection{简述 PLL 的作用?}

当芯片工作频率高于一定频率时,就需要消除由于芯片内时钟驱动所引起的片内时钟与片外时钟间的相位差,嵌入在芯片内部的PLL可以消除这种时钟延时。

此外,很多芯片控制链逻辑需要占空比为50\%的时钟,因此需要一个2倍于此的时钟源,集成在芯片内部的PLL可以将外部时钟合成为此时钟源。最后,系统集成PLL可以从内部触发,比从外部触发更快且更准确,能有效地避免一些与信号完整性相关的问题。


\setcounter{subsection}{5}
\subsection{什么是中断?简述 CPU 响应中断的处理过程。}

计算机在执行程序的过程中,当出现了某个特殊事件时,CPU 会终止当前程序的执行,转而去执行该事件的处理程序(中断服务程序),待中断服务程序执行完毕,在返回断点继续执行原来的程序,这个过程称为中断。

中断处理过程:中断请求 $\rightarrow$ 判断该中断是否被屏蔽 $\rightarrow$ 保护现场 $\rightarrow$ 查找中断服务程序地址 $\rightarrow$ 执行中断服务程序 $\rightarrow$ 中断返回


\setcounter{subsection}{8}
\subsection{说明 ARM Cortex-M4 嵌套向量中断控制器(NVIC)的基本功能。}

支持异常中断;支持可编程异常优先级;中断响应时处理器状态的自动保存;中断返回时处理器状态的自动恢复;支持嵌套和向量中断。



\setcounter{subsection}{9}
\subsection{什么是 DMA?DMA 传输是以总线周期还是指令周期为单位的}

DMA是在主存和外设之间开辟直接的数据通路,可以进行基本上不需要CPU介入的主存和外设之间的信息传送,这样不仅能保证CPU的高效率,而且能满足高速外设的需要。DMA传输是以总线周期为单位。


\setcounter{subsection}{10}
\subsection{CPU与外设进行数据传输一般有哪几种方式?各自的特点和适用性?}

方式有四种:程序查询方式、中断控制方式、DMA方式、通道方式。
程序查询方式是一种程序直接控制方式,这是主机与外设间进行信息交换的最简单方式,输入和输出完全是通过CPU执行程序来完成的。一旦某一外设被选中并启动之后,主机将查询这个外设的某些状态位,看其是否准备就绪?若外设未准备就绪,主机将再次查询;若外设已准备就绪,则执行一次I/O操作。 
中断控制方式,在主机启动外设后,无须等待查询,而是继续执行原来的程序,外设在做好输入输出准备时,向主机发中断请求,主机接到请求后就暂时中止原来执行的程序,转去执行中断服务程序对外部请求进行处理,在中断处理完毕后返回原来的程序继续执行。
DMA方式是在主存和外设之间开辟直接的数据通路,可以进行基本上不需要CPU介入的主存和外设之间的信息传送,这样不仅能保证CPU的高效率,而且能满足高速外设的需要。DMA方式只能进行简单的数据传送操作,在数据块传送的起始和结束时还需CPU及中断系统进行预处理和后处理通道方式,与DMA方式相类似,也是一种以内存为中心,实现设备和内存直接交换数据的控制方式。
通道控制方式是DMA方式的进一步发展,在系统中设有通道控制部件,每个通道挂若干外设,主机在执行I/O操作时,只需启动有关通道,通道将执行通道程序,从而完成I/O操作。通道是一个具有特殊功能的处理器,它能独立地执行通道程序,产生相应的控制信号,实现对外设的统一管理和外设与主存之间的数据传送。但它不是一个完全独立的处理器。它要在CPU的I/O指令指挥下才能启动、停止或改变工作状态,是从属于CPU的一个专用处理器。



\setcounter{subsection}{12}
\subsection{简述嵌入式系统电源/功耗控制的意义?以 MSP432 为例,说明其电源/功耗控制功能。}

MSP432微控制器新增了包括双稳压器(LDO和DC-DC)在内的一系列电源系统功能,可为内部逻辑和其他组件生成两个VCORE电平。内核电压频率限制和运行时可选稳压器这类功能可提高器件的灵活性,从而可进一步优化电源系统以及系统功耗。 

MSP432P微控制器的低频功耗模式能够将最大工作频率限制在 128kHz,以此实现非常低的功耗。

MSP432 MCU 除了工作模式之外,还提供多种低功耗模式,各模式下会对不同时钟和外设的功率进行门控,从而可针对不同应用状态下的功耗灵活地进行优化。





\end{document}
