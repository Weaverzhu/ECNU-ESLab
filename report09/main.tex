% !TEX encoding = UTF-8 Unicode
\RequirePackage{fix-cm}
\documentclass[a4paper,10pt,UTF8]{paper}
%\documentclass[a4paper,10pt,UTF8]{ctexart}

\usepackage[english]{babel}
\usepackage{fancyhdr,array,lastpage,amsmath,mathtools,enumitem,graphicx,multirow,tocbibind,longtable,makecell,varwidth,titlesec,bm,booktabs,comment,minted}
\usepackage{enumitem}
\usepackage{hyperref}
\hypersetup{hidelinks}




\usepackage[left=2.54cm,right=2.54cm,top=2.54cm,bottom=2.54cm]{geometry}
\usepackage[font=footnotesize,labelfont=bf]{caption}
\usepackage{tikz,flowchart}
\usepackage{ctex}
\usepackage{xeCJK}%中文字体
\usetikzlibrary{shapes,shapes.geometric,arrows,matrix,calc}
\usetikzlibrary{circuits.logic}

% \usetikzlibrary{circuits.logic.custom}
\usetikzlibrary{circuits.logic.IEC}
\usetikzlibrary{shadows}
\usepackage{listings}
\usepackage[Q=yes]{examplep}
\usepackage{fancyhdr}
\usepackage{alphalph}
\usepackage{indentfirst}

% \setCJKsansfont{黑体}
\setmainfont{PingFang SC}
\setCJKmainfont{PingFang SC}
\setCJKsansfont{PingFang SC}
\setmonofont{Monaco}

\newenvironment{sol}
  {\par\vspace{2mm}\noindent{\bf Solution}. }

\lstset{escapeinside=``, breaklines=true, frame=none, extendedchars=false, basicstyle=\ttfamily, showstringspaces=false}


\setlength{\parindent}{2em}
\setlength{\parskip}{1.5ex plus 0.5ex minus 0.2ex}
\linespread{1.1}

\bibliographystyle{plain}

\numberwithin{equation}{section}
\numberwithin{figure}{section}


\setcounter{secnumdepth}{3}
\setcounter{tocdepth}{3}

\title{华东师范大学计算机科学技术系上机实验报告}

\begin{document}
\pagestyle{fancy}
\chead{\small\color{gray}华东师范大学计算机科学技术系上机实验报告}
\lhead{}
\rhead{}
\makeatletter
\def\headrule{{\if@fancyplain\let\headrulewidth\plainheadrulewidth\fi%
\color{gray}\hrule\@height 0.2pt\@width\headwidth}
  \vspace{6mm}}
\makeatother

\newcommand{\HRule}{\rule{\linewidth}{1mm}}
\newcommand{\dai}{\textbf{Dais-CMX16$^+$}}

{\center {\huge \bfseries \LARGE{华东师范大学计算机科学技术系上机实验报告}} \\ [0.8cm]

\small{
  \begin{minipage}[t]{.32\linewidth}
    \textbf{课程名称:}嵌入式系统原理与实践\\
    \textbf{指导教师:}沈建华\\
    \textbf{上机实践名称:} 脉冲宽度调制\\
    \textbf{实践编号:}实验 9
  \end{minipage}
  \begin{minipage}[t]{.32\linewidth}
    \textbf{年级:}17 级\\
    \textbf{姓名:}朱桐\\
    \textbf{学号:}10175102111\\
    \textbf{组号:}A
  \end{minipage} 
  \begin{minipage}[t]{.32\linewidth}
    \textbf{上机实践成绩:} \\
    \textbf{创新实践成绩:} \\
    \textbf{上机实践日期:}2019/09/20\\
    \textbf{上机实践时间:}2 学时\\
  \end{minipage}
}
\HRule \\[0.5cm]
}

\definecolor{bg}{rgb}{0.95,0.95,0.95}
\newminted{asm}{bgcolor=bg}
\newminted{c}{bgcolor=bg}
\section{实验目的}

\begin{enumerate}
    \item 理解中断的基本原理,掌握利用定时器中断进行精确定时的方法
    \item 理解PWM的基本原理,掌握编程实现PWM的方法
\end{enumerate}

\section{实验设备}


\begin{enumerate}
  \item 软件Keil5
  \item MSP-EXP432P401R开发板

\end{enumerate}

\section{实验内容}

\begin{enumerate}
  \item 使用MSP432的TimerA模块产生定时中断,观察实验现象
  \item 使用MSP432的TimerA模块产生PWM,观察实验现象
  \item 完成课堂练习:多彩灯的实现和台灯的完善。
\end{enumerate}

\section{实验原理}

\subsection{PWM}

脉冲宽度调制是一种对模拟信号电平的数字编码方法,被广泛应用在测量、通信、功率控制和变换等许多领域。对单片机而言,PWM可以用定时器的输出比较功能实现;在MSP432中,我们可以使用TimerA实现PWM。


\subsection{TimerA}

MSP432 中一共有 4 个 TimerA 模块,每个模块包含多达7个捕获/比较寄存器。Timer\_A有间隔定时功能和PWM输出功能,同时还具有广泛的中断功能。中断在计数器溢出时产生,也可以由比较/捕获寄存器产生。

TimerA 时钟可以由 ACLK、 SMCLK 或外部 TAxCLK 和 INCLK产生,时钟源可以直接传递给定时器或者进行时钟分频后传递给定时器供定时器使用。定时器有 4 种计数模式:连续计数模式、增计数模式、增减计数模式和停止。

我们设计实现了一个简单的灯闪烁功能,时钟源使用DCO时钟,采用MSP432 LaunchPad的TimerA0模块和增计数模式,来控制灯闪烁间隔时间的精确控制,而前面通过时钟和延时函数的配合实现的定时间隔与之相比较而言,只是通过计算得到的一个大概的估算值。

\subsection{函数介绍}

\begin{itemize}
  \item \textbf{void Timer\_A\_configureUpMode(uint32\_t timer,const Timer\_A\_UpModeConfig *config);}
  \item 功能描述:Timer\_A配置增计数模式
  \item 参数描述:timer   指定定时器, config  指定增计数模式配置参数
\end{itemize}

\begin{itemize}
  \item \textbf{void Timer\_A\_startCounter(uint32\_t timer,uint\_fast16\_t timerMode);   }
  \item 功能描述:Timer\_A开始计数
  \item 参数描述:timer   指定定时器, timerMode  指定定时器模式
\end{itemize}

\section{实验步骤}

\subsection{实验1:TimerA定时中断实验}

\begin{enumerate}
  \item 打开位于工程09\_01的timerA工程;
  \item 点击编译并下载至LaunchPad开发板,观察并描述LED1(红灯)变化情况(说明闪烁时间间隔具体几秒)。
\end{enumerate}

每个一秒切换一次亮暗状态,每隔两秒亮一秒

\subsection{实验2:TimerA输出PWM(单通道)}

\begin{enumerate}
  \item 打开位于工程09\_02的timerA\_single\_pwm工程;
  \item 点击编译并下载至LaunchPad开发板,观察并描述LED2的绿灯变化情况(具体说明LED2改变了几个亮度级)。
\end{enumerate}

共十个亮度级别,从亮变暗到暗,再变亮到最亮,如此重复

\subsection{实验3:TimerA输出PWM(多通道)}

\begin{enumerate}
  \item 打开位于工程09\_03的timerA\_multiply\_pwm工程;
  \item 点击编译并下载至LaunchPad开发板,观察并描述LED2变化情况(具体说明LED2状态变化的时间间隔)。
\end{enumerate}

每隔一秒,RGB三个通道的亮度随机发生改变,所以颜色和亮度每秒随机发生改变

\subsection{课堂练习1:多彩灯}

\begin{enumerate}
  \item 新建交通信号灯keil工程,工程名为:color\_学号,并保存到工程09\_04文件夹下;
  \item 请让LED2实现10种以上的颜色,每种颜色持续时间1秒。
\end{enumerate}

\subsection{课堂练习2:台灯的完善}

\begin{enumerate}
  \item 新建交通信号灯keil工程,工程名为:light2\_学号,并保存到工程09\_05文件夹下;
  \item 使用学号后三位作为三个不同的亮度级(0最暗,9最亮),控制LED2亮白色并以这三种亮度循环闪烁,每种亮度保持时间为1秒(比如学号后三位为013,那么分别使用0,1,3三个数字来设置3个不同大小的占空比)。
\end{enumerate}

\section{调试过程、结果与分析}

使用 RGB 三元组可控制色调和亮度。使用占空比反应 RGB 的值,占空比越高颜色越明显

\subsection{课堂练习1}

自定义颜色数组,使用 \texttt{curcolor(0-9)} 表示颜色状态。使用 \textbf{MAP\_Timer\_A\_initCompare} 控制表示

\begin{ccode}
  int curcolor;
  const int color[10][3] = {
    {0,0,255},
    {0,255,0},
    {255,0,0},
    {255,90,40},
    {0,255,255},
    {255,0,255},
    {127,60,40},
    {255,255,0},
    {255,255,255},
    {127,40,255}
  };
  void TA1_0_IRQHandler(void)
  {
    curcolor = (curcolor+1) % TOTAL;
    TA0_CCR1_PWM.compareValue = PWM_PERIODS * color[curcolor][0] / 255;
    MAP_Timer_A_initCompare(TIMER_A0_BASE, &TA0_CCR1_PWM); 

    TA0_CCR2_PWM.compareValue = PWM_PERIODS * color[curcolor][1] / 255;
    MAP_Timer_A_initCompare(TIMER_A0_BASE, &TA0_CCR2_PWM); 

    TA0_CCR3_PWM.compareValue = PWM_PERIODS * color[curcolor][2] / 255;
    MAP_Timer_A_initCompare(TIMER_A0_BASE, &TA0_CCR3_PWM); 

    MAP_Timer_A_clearCaptureCompareInterrupt(TIMER_A1_BASE, 
    TIMER_A_CAPTURECOMPARE_REGISTER_0);
  }
\end{ccode}

\subsection{课堂练习2}

如果需要亮白灯,那么三个通道的强度必须一致

和之前一样,用 \texttt{state} 表示学号位,由于我三位学号都是一样的。。所以看不出改变,但是如果把学号改成1,5,9也是可以看到明显的改变的。

\begin{ccode}
  int state;
  const int stuno[] = {
      1, 1, 1
  };

  void TA1_0_IRQHandler(void)
  {
    state = (state + 1) % 3;
    TA0_CCR1_PWM.compareValue = PWM_PERIODS * stuno[state] / 10;
    MAP_Timer_A_initCompare(TIMER_A0_BASE, &TA0_CCR1_PWM); 

    TA0_CCR2_PWM.compareValue = PWM_PERIODS * stuno[state] / 10;
    MAP_Timer_A_initCompare(TIMER_A0_BASE, &TA0_CCR2_PWM); 

    TA0_CCR3_PWM.compareValue = PWM_PERIODS * stuno[state] / 10;
    MAP_Timer_A_initCompare(TIMER_A0_BASE, &TA0_CCR3_PWM); 

    MAP_Timer_A_clearCaptureCompareInterrupt(TIMER_A1_BASE, 
    TIMER_A_CAPTURECOMPARE_REGISTER_0); 
  }
\end{ccode}

\section{总结}

\begin{ccode}
  typedef struct _Timer_A_UpModeConfig
  {
    uint_fast16_t clockSource;
    uint_fast16_t clockSourceDivider;
    uint_fast16_t timerPeriod;
    uint_fast16_t timerInterruptEnable_TAIE;
    uint_fast16_t captureCompareInterruptEnable_CCR0_CCIE;
    uint_fast16_t timerClear;
  } Timer_A_UpModeConfig;
\end{ccode}

\begin{enumerate}
  \item 别忘了清除中断标志,否则现象会是一直处于中断状态
  \item 留心查看函数参数,\texttt{PWM\_period} 的数据类型是 \texttt{uint\_fast16\_t},过高的数值会导致越界
\end{enumerate}

\section{附件}

\end{document}
