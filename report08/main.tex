% !TEX encoding = UTF-8 Unicode
\RequirePackage{fix-cm}
\documentclass[a4paper,10pt,UTF8]{paper}
%\documentclass[a4paper,10pt,UTF8]{ctexart}

\usepackage[english]{babel}
\usepackage{fancyhdr,array,lastpage,amsmath,mathtools,enumitem,graphicx,multirow,tocbibind,longtable,makecell,varwidth,titlesec,bm,booktabs,comment,minted}
\usepackage{enumitem}
\usepackage{hyperref}
\hypersetup{hidelinks}




\usepackage[left=2.54cm,right=2.54cm,top=2.54cm,bottom=2.54cm]{geometry}
\usepackage[font=footnotesize,labelfont=bf]{caption}
\usepackage{tikz,flowchart}
\usepackage{ctex}
\usepackage{xeCJK}%中文字体
\usetikzlibrary{shapes,shapes.geometric,arrows,matrix,calc}
\usetikzlibrary{circuits.logic}

% \usetikzlibrary{circuits.logic.custom}
\usetikzlibrary{circuits.logic.IEC}
\usetikzlibrary{shadows}
\usepackage{listings}
\usepackage[Q=yes]{examplep}
\usepackage{fancyhdr}
\usepackage{alphalph}
\usepackage{indentfirst}

% \setCJKsansfont{黑体}
\setmainfont{PingFang SC}
\setCJKmainfont{PingFang SC}
\setCJKsansfont{PingFang SC}
\setmonofont{Monaco}

\newenvironment{sol}
  {\par\vspace{2mm}\noindent{\bf Solution}. }

\lstset{escapeinside=``, breaklines=true, frame=none, extendedchars=false, basicstyle=\ttfamily, showstringspaces=false}


\setlength{\parindent}{2em}
\setlength{\parskip}{1.5ex plus 0.5ex minus 0.2ex}
\linespread{1.1}

\bibliographystyle{plain}

\numberwithin{equation}{section}
\numberwithin{figure}{section}


\setcounter{secnumdepth}{3}
\setcounter{tocdepth}{3}

\title{华东师范大学计算机科学技术系上机实验报告}

\begin{document}
\pagestyle{fancy}
\chead{\small\color{gray}华东师范大学计算机科学技术系上机实验报告}
\lhead{}
\rhead{}
\makeatletter
\def\headrule{{\if@fancyplain\let\headrulewidth\plainheadrulewidth\fi%
\color{gray}\hrule\@height 0.2pt\@width\headwidth}
  \vspace{6mm}}
\makeatother

\newcommand{\HRule}{\rule{\linewidth}{1mm}}
\newcommand{\dai}{\textbf{Dais-CMX16$^+$}}

{\center {\huge \bfseries \LARGE{华东师范大学计算机科学技术系上机实验报告}} \\ [0.8cm]

\small{
  \begin{minipage}[t]{.32\linewidth}
    \textbf{课程名称:}计算机组成与结构实践\\
    \textbf{指导教师:}金健\\
    \textbf{上机实践名称:} TODO\\
    \textbf{实践编号:}实验 1
  \end{minipage}
  \begin{minipage}[t]{.32\linewidth}
    \textbf{年级:}17 级\\
    \textbf{姓名:}朱桐\\
    \textbf{学号:}10175102111\\
    \textbf{组号:}A
  \end{minipage} 
  \begin{minipage}[t]{.32\linewidth}
    \textbf{上机实践成绩:} \\
    \textbf{创新实践成绩:} \\
    \textbf{上机实践日期:}2019/09/20\\
    \textbf{上机实践时间:}2 学时\\
  \end{minipage}
}
\HRule \\[0.5cm]
}

\definecolor{bg}{rgb}{0.95,0.95,0.95}
\newminted{asm}{bgcolor=bg}

\section{实验目的}

\begin{enumerate}
    \item 了解两种控制方式:“手动”和“微控制”
    \item 了解两种实验方式:“搭接”和“在线”
    \item 理解并学会设置手动搭接实验方式
    \item 熟悉与了解准双向 I/O 口的构成原理
    \item 掌握准双向 I/O 口的输入/输出特性的运用
    \item 熟悉通用寄存器的数据通路
    \item 掌握通用寄存器的构成和运用
\end{enumerate}

\section{实验设备}


\section{实验内容}


\section{实验原理}

\section{实验步骤}

\section{调试过程、结果与分析}

\section{总结}

\section{附件}

\end{document}
