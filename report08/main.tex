% !TEX encoding = UTF-8 Unicode
\RequirePackage{fix-cm}
\documentclass[a4paper,10pt,UTF8]{paper}
%\documentclass[a4paper,10pt,UTF8]{ctexart}

\usepackage[english]{babel}
\usepackage{fancyhdr,array,lastpage,amsmath,mathtools,enumitem,graphicx,multirow,tocbibind,longtable,makecell,varwidth,titlesec,bm,booktabs,comment,minted}
\usepackage{enumitem}
\usepackage{hyperref}
\hypersetup{hidelinks}




\usepackage[left=2.54cm,right=2.54cm,top=2.54cm,bottom=2.54cm]{geometry}
\usepackage[font=footnotesize,labelfont=bf]{caption}
\usepackage{tikz,flowchart}
\usepackage{ctex}
\usepackage{xeCJK}%中文字体
\usetikzlibrary{shapes,shapes.geometric,arrows,matrix,calc}
\usetikzlibrary{circuits.logic}

% \usetikzlibrary{circuits.logic.custom}
\usetikzlibrary{circuits.logic.IEC}
\usetikzlibrary{shadows}
\usepackage{listings}
\usepackage[Q=yes]{examplep}
\usepackage{fancyhdr}
\usepackage{alphalph}
\usepackage{indentfirst}

% \setCJKsansfont{黑体}
\setmainfont{PingFang SC}
\setCJKmainfont{PingFang SC}
\setCJKsansfont{PingFang SC}
\setmonofont{Monaco}

\newenvironment{sol}
  {\par\vspace{2mm}\noindent{\bf Solution}. }

\lstset{escapeinside=``, breaklines=true, frame=none, extendedchars=false, basicstyle=\ttfamily, showstringspaces=false}


\setlength{\parindent}{2em}
\setlength{\parskip}{1.5ex plus 0.5ex minus 0.2ex}
\linespread{1.1}

\bibliographystyle{plain}

\numberwithin{equation}{section}
\numberwithin{figure}{section}


\setcounter{secnumdepth}{3}
\setcounter{tocdepth}{3}

\title{华东师范大学计算机科学技术系上机实验报告}

\begin{document}
\pagestyle{fancy}
\chead{\small\color{gray}华东师范大学计算机科学技术系上机实验报告}
\lhead{}
\rhead{}
\makeatletter
\def\headrule{{\if@fancyplain\let\headrulewidth\plainheadrulewidth\fi%
    \color{gray}\hrule\@height 0.2pt\@width\headwidth}
    \vspace{6mm}}
\makeatother

\newcommand{\HRule}{\rule{\linewidth}{1mm}}
\newcommand{\dai}{\textbf{Dais-CMX16$^+$}}

{\center {\huge \bfseries \LARGE{华东师范大学计算机科学技术系上机实验报告}} \\ [0.8cm]
    
    \small{
        \begin{minipage}[t]{.32\linewidth}
            \textbf{课程名称:}嵌入式系统原理与实践\\
            \textbf{指导教师:}沈建华\\
            \textbf{上机实践名称:} 基本定时器\\
            \textbf{实践编号:}实验 8
        \end{minipage}
        \begin{minipage}[t]{.32\linewidth}
            \textbf{年级:}17 级\\
            \textbf{姓名:}朱桐\\
            \textbf{学号:}10175102111\\
            \textbf{组号:}A
        \end{minipage} 
        \begin{minipage}[t]{.32\linewidth}
            \textbf{上机实践成绩:} \\
            \textbf{创新实践成绩:} \\
            \textbf{上机实践日期:}2019/11/26\\
            \textbf{上机实践时间:}2 学时\\
        \end{minipage}
    }
    \HRule \\[0.5cm]
}

\definecolor{bg}{rgb}{0.95,0.95,0.95}
\newminted{asm}{bgcolor=bg}
\newminted{c}{bgcolor=bg}

\section{实验目的}

\begin{enumerate}
    \item 掌握基本定时器Timer32相关库函数的配置
    \item 学会使用定时器进行定时并触发中断
\end{enumerate}

\section{实验设备}

\begin{enumerate}
    \item 软件Keil5
    \item MSP-EXP432P401R开发板
    \item LED交通信号灯模块
\end{enumerate}

\section{实验内容}

\begin{enumerate}
    \item 理解定时器的原理,完成更精确的灯闪练习
    \item 利用定时器完善交通信号灯实验
\end{enumerate}

\section{实验原理}

\subsection{基本定时器}

MSP432包含多种定时器,比如Timer32、TimerA、WDT(看门狗定时器)Systick(系统滴答定时器)。其中Timer32定时器由两个可配置的计数器组成,当计数器递减到0时产生中断。其具有以下几个特点:

\begin{enumerate}
    \item 支持两个独立的计数器,每个计数器可配置为32位模式或16位模式;
    \item 每个计数器支持三种不同的计时模式:自由计数、周期计数和单次计数;
    \item 可配置分频系数,可将输入时钟频率除以1、16或256;
    \item 支持单个计数器独立中断,同样也支持两个计数器的组合中断。
\end{enumerate}

设置 DCO 和 MCLK 频率

\begin{ccode}
    MAP_CS_setDCOFrequency(10000000); /* 设置DCO频率为指定频率,此处DCO=10M*/
    //![Simple FPU Config]
    
    MAP_CS_initClockSignal(CS_MCLK, CS_DCOCLK_SELECT, CS_CLOCK_DIVIDER_1); 
    /*设置MCLK(主时钟频率),MCLK=DCO频率/时钟分频系数,此处MCLK=DCO=1M*/
\end{ccode}

设置时钟中断和周期,这里将时钟周期和频率设置一致,则周期为 1s。秒数 = 周期 / 频率

\begin{ccode}
    MAP_Timer32_initModule(TIMER32_0_BASE, TIMER32_PRESCALER_1, 
    TIMER32_32BIT,TIMER32_PERIODIC_MODE);
    
    MAP_Interrupt_enableInterrupt(INT_T32_INT1);    
    MAP_Timer32_enableInterrupt(TIMER32_0_BASE);
    
    MAP_Interrupt_enableMaster();/*使能中断总开关*/
    
    MAP_Timer32_setCount(TIMER32_0_BASE,10000000);
\end{ccode}

中断函数,别忘了清除中断标志

\begin{ccode}
    void T32_INT1_IRQHandler(void)
    {
        MAP_Timer32_clearInterruptFlag(TIMER32_0_BASE); 
    }
\end{ccode}

\section{实验步骤}

\subsection{练习1}

\begin{enumerate}
    \item 请将工程08\_01中的MCLK配置为10M;
    \item 请使用学号后两位数字配置LED的闪烁周期:例如学号后两位数字为09,即0和9(若有数字为0,请转成10),则让LED1的红灯以10秒的间隔进行闪烁,同时让LED2的绿灯以9秒的间隔进行闪烁。
\end{enumerate}


\subsection{练习2}

\begin{enumerate}
    \item 新建交通信号灯keil工程,工程名为:tsl2\_学号,并保存到工程08\_02文件夹下;
    \item 上电运行后,信号灯处于循环模式,即按照绿灯(亮5秒)、黄灯(亮1.5秒)、红灯(亮2.5秒)的顺序依次循环;
    \item 按压S1键,信号灯的绿灯一直保持常亮,表示允许通过;
    \item 按压S2键,信号灯的红灯一直保持常亮,表示禁止通过;
    \item 当信号灯处于通行或者禁行模式时,同时按压S1键和S2键,信号灯回到循环模式(从绿灯开始循环);
    \item 请使用新的LED交通信号灯模块来实现。
\end{enumerate}

\section{调试过程、结果与分析}

\subsection{练习1}


更改周期即可,非常简单。学号10175102111,这里把两个周期都设置为 1s

\begin{ccode}
    MAP_CS_setDCOFrequency(10000000); /* 设置DCO频率为指定频率,此处DCO=10M*/
    //![Simple FPU Config]
    
    MAP_CS_initClockSignal(CS_MCLK, CS_DCOCLK_SELECT, CS_CLOCK_DIVIDER_1); 
    /*设置MCLK(主时钟频率),MCLK=DCO频率/时钟分频系数,此处MCLK=DCO=1M*/

    MAP_Timer32_setCount(TIMER32_0_BASE,10000000);
    MAP_Timer32_setCount(TIMER32_1_BASE,10000000);
\end{ccode}

\subsection{练习2}

\begin{enumerate}
    \item 使用 0-2 三个整数表示红绿灯三种颜色的状态
    \item 使用计时器计时,计时器中断的时候切换状态
    \item 使用另一个计时器计时,按下第一个键时候开始计时,等待一段时间后如果按下第二个键,视为两个键一切按下,否则算作一下一个键(消噪)
\end{enumerate}

初始化 MCLK, DCO

\begin{ccode}
    MAP_WDT_A_holdTimer();
    MAP_FPU_enableModule();
    MAP_CS_setDCOFrequency(10000000);

    MAP_CS_initClockSignal(CS_MCLK, CS_DCOCLK_SELECT, CS_CLOCK_DIVIDER_1);
\end{ccode}

设置 GPIO,这里将 RYG 分别设置为 P4.0, P4.0, P5.4

\begin{ccode}
    MAP_GPIO_setAsOutputPin(GPIO_PORT_P1, GPIO_PIN0);
    MAP_GPIO_setOutputHighOnPin(GPIO_PORT_P1, GPIO_PIN0);

    MAP_GPIO_setAsOutputPin(GPIO_PORT_P4, GPIO_PIN0);
    MAP_GPIO_setOutputHighOnPin(GPIO_PORT_P4, GPIO_PIN0);

    MAP_GPIO_setAsOutputPin(GPIO_PORT_P4, GPIO_PIN4);
    MAP_GPIO_setOutputHighOnPin(GPIO_PORT_P4, GPIO_PIN4);

    MAP_GPIO_setAsOutputPin(GPIO_PORT_P5, GPIO_PIN4);
    MAP_GPIO_setOutputHighOnPin(GPIO_PORT_P5, GPIO_PIN4);
    
    MAP_GPIO_setAsInputPinWithPullUpResistor(GPIO_PORT_P1, GPIO_PIN1);
    MAP_GPIO_setAsInputPinWithPullUpResistor(GPIO_PORT_P1, GPIO_PIN4);

    MAP_GPIO_enableInterrupt(GPIO_PORT_P1, GPIO_PIN1);
    MAP_GPIO_clearInterruptFlag(GPIO_PORT_P1, GPIO_PIN1);

    MAP_GPIO_enableInterrupt(GPIO_PORT_P1, GPIO_PIN4);
    MAP_GPIO_clearInterruptFlag(GPIO_PORT_P1, GPIO_PIN4);

    MAP_Interrupt_enableInterrupt(INT_PORT1);
    MAP_Interrupt_enableMaster();

    MAP_GPIO_setOutputLowOnPin(GPIO_PORT_P1, GPIO_PIN0);
    MAP_GPIO_setOutputLowOnPin(GPIO_PORT_P5, GPIO_PIN4);
    MAP_GPIO_setOutputLowOnPin(GPIO_PORT_P4, GPIO_PIN4);
    MAP_GPIO_setOutputLowOnPin(GPIO_PORT_P4, GPIO_PIN0);
\end{ccode}

初始化时钟

\begin{ccode}
    MAP_Timer32_initModule(TIMER32_0_BASE, TIMER32_PRESCALER_1,
     TIMER32_32BIT, TIMER32_PERIODIC_MODE);
    MAP_Timer32_initModule(TIMER32_1_BASE, TIMER32_PRESCALER_1,
     TIMER32_32BIT, TIMER32_PERIODIC_MODE);

    MAP_Interrupt_enableInterrupt(INT_T32_INT1);
    MAP_Timer32_enableInterrupt(TIMER32_0_BASE);

    MAP_Interrupt_enableInterrupt(INT_T32_INT2);
    MAP_Timer32_enableInterrupt(TIMER32_1_BASE);
\end{ccode}

初始化颜色为绿色,并开启主循环

\begin{ccode}
    setcolor(color);
    while (1)
    {
    }
\end{ccode}

第一个时钟用来切换颜色

\begin{ccode}
    void T32_INT1_IRQHandler(void)
    {
        MAP_Timer32_clearInterruptFlag(TIMER32_0_BASE);
        switchcolor();
    }
\end{ccode}

切换颜色的函数,切换颜色的时候重新设置周期并计时

\begin{ccode}
    const int duration[] = {
    60000000, 15000000, 25000000};

    void setcolor(int color)
    {
        MAP_GPIO_setOutputLowOnPin(GPIO_PORT_P5, GPIO_PIN4);
        MAP_GPIO_setOutputLowOnPin(GPIO_PORT_P4, GPIO_PIN4);
        MAP_GPIO_setOutputLowOnPin(GPIO_PORT_P4, GPIO_PIN0);
        switch (color)
        {
        case 0:
            MAP_GPIO_setOutputHighOnPin(GPIO_PORT_P5, GPIO_PIN4);
            break;
        case 1:
            MAP_GPIO_setOutputHighOnPin(GPIO_PORT_P4, GPIO_PIN4);
            break;
        case 2:
            MAP_GPIO_setOutputHighOnPin(GPIO_PORT_P4, GPIO_PIN0);

        default:
            break;
        }
        MAP_Timer32_setCount(TIMER32_0_BASE, duration[color]);
        MAP_Timer32_startTimer(TIMER32_0_BASE, false);
    }
\end{ccode}

第二个时钟配合 GPIO P1.1, P1.4 输入,判断只按 S1, S2 或者两者一起按下

\begin{ccode}
    bool flag1, flag2;
    int color, condition;
\end{ccode}

\begin{ccode}
    void PORT1_IRQHandler(void)
    {
        uint32_t status;
        status = MAP_GPIO_getEnabledInterruptStatus(GPIO_PORT_P1);
        MAP_GPIO_clearInterruptFlag(GPIO_PORT_P1, status);

        if (status & GPIO_PIN1)
        {
            if (condition == 0)
            {
                MAP_Timer32_setCount(TIMER32_1_BASE, 1000000);

                MAP_GPIO_toggleOutputOnPin(GPIO_PORT_P1, GPIO_PIN0);
                MAP_Timer32_startTimer(TIMER32_1_BASE, false);
                condition = 1;
            }
            else
            {
                flag1 = flag2 = 0;
                setcolor(color = 0);
                condition = 0;
            }
        }
        else if (status & GPIO_PIN4)
        {

            if (condition == 0)
            {
                MAP_Timer32_setCount(TIMER32_1_BASE, 1000000);
                MAP_Timer32_startTimer(TIMER32_1_BASE, false);
                MAP_GPIO_toggleOutputOnPin(GPIO_PORT_P1, GPIO_PIN0);
                condition = 2;
            }
            else
            {
                flag1 = flag2 = 0;
                setcolor(color = 0);
                condition = 0;
            }
        }
        else if ((status & GPIO_PIN1) && (status & GPIO_PIN4))
        {
            flag1 = flag2 = 0;
            setcolor(color = 0);
            condition = 0;
        }
    }

    void T32_INT2_IRQHandler(void)
    {
        MAP_Timer32_clearInterruptFlag(TIMER32_1_BASE);
        MAP_GPIO_toggleOutputOnPin(GPIO_PORT_P1, GPIO_PIN0);
        if (condition == 1)
        {
            flag1 = 1;
            flag2 = 0;
            color = 0;
            setcolor(color);
        }
        else if (condition == 2)
        {
            flag1 = 0;
            flag2 = 1;
            color = 2;
            setcolor(color);
        }
        condition = 0;
    }
\end{ccode}

\section{总结}

实际上实现的功能前几次都做过了,只是把空循环换成时钟中断。不过时钟中断的优点是不会影响程序控制流,否则可能会造成输入信号丢失。

\section{附件}

\end{document}
